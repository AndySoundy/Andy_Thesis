\documentclass[12pt]{article}

\usepackage[english]{babel}
\usepackage{amsmath}
\usepackage{graphicx}
\usepackage{indentfirst}

\title{Literature Review of Magnetoreception in Elasmobranches}
\author{Andy Soundy}

\begin{document}
\maketitle

\section{Introduction}

It has long been known that sharks are pointy and bitey, what is less well known is their ability to detect the earth's magnetic field. Many species of elasmobranches undertake long-distance migrations for feeding, breeding and other reasons\cite{Stevens:1976}\cite{Carey:1990}\cite{Bonfil:2005}. Some of these migrations cover extremely large distances through pelagic enviroments with few or no geographic landmarks with which the animals may orient themselves\cite{Bonfil:2005}. The earth's magnetic field would be one cue that could be used by the animals to orient themselves with. Navigating by the magnetic field has several advantages for long-distance marine migrations, it is a stable cue that is not confined to a specific area of the ocean, it is relatively constant when compared with cues such as temperature fluctuations or currents and if an induction-based method for detecting the magnetic field is accurate then the animal would even be able to correct for current drifts\cite{Paulin:1995}.  

\section{Magnetoreception in animals}
\label{sec:examples}


Several experiments have shown that at least some elasmobranch species may be conditioned to respond to the presence or absence of magnetic fields\cite{Kalmijn:1982}\cite{Hodson:2000}\cite{Walker:2003}. While there is little debate as to whether or not elasmobranches can detect magnetic fields there is considerably more as to the mechansim by which they detect magnetic fields. In general there are three main mechanisms by which it is proposed that animals may detect magnetic fields\cite{Johnsen:2008}. The first is that there exist chains of ferri-magnetic crystals (likely magnetite or some derivative thereof) located within the animal. These chains align with or precess about the direction of the magnetic field and the animals use the information from the movement of the crystals to navigate. Another theory is that there exists a chemical located in the visual system that uses a chemical reaction to determine the direction of the magnetic field. There is a class of chemical reactions involving radical pairs which are very receptive to the strength and direction of magnetic fields\cite{Grissom:1995}\cite{Ritz:2004}, and from the reactions of these radical pairs the animal may be able to determine a compass heading\cite{Johnsen:2008}\cite{Ritz:2000}. Finally there is the theory of induction-based magnetoreception\cite{Paulin:1995}\cite{Kalmijn:1982}. Elasmobranches are able incredibly sensitive to electric fields via their ampullae of Lorenzini. These ampullae resemble canals filled with a conductive fluid, at the ends of which are receptors that are able to detect voltages. As the elasmobranch swims through the earth's magnetic field it induces a voltage which the ampullae detect and the animal then navigates via this induced voltage.

\section{Magnetoreception Mechanisms}
\subsection{Magnetite Crystals}

In order for animals to navigate by using chains of magnetite crystals inside cells in the animals there are a few barriers that must be overcome. First at the cellular level the interaction energy of the crystal chains with the earth's magnetic field must overcome the randomising effect of background thermal energy (\textit{kT}) where \textit{k} is the Boltzmann constant. The interaction energy between the magnetite crystals and the earth's magnetic field will be \textit{$\mu B$} where \textit{$\mu$} is the magnetic moment and \textit{B} is the strength of the earth's magnetic field\cite{Kirschvink:2001}.  Of the three proposed mechanisms the theory of magnetite crystals being used to detect magnetic fields certainly seem to have to have the strongest evidence, at least for animals outside the elasmobranch subclass. Despite the small size of the magnetite crystals themselves ($<=$50nm) and extreme rarity (\textless5p.p.b. by volume) long chains of magnetite crystals have been successfully identified in rainbow trout (\textit{Oncorhynchus mykiss}), homing pigeons (\textit{Columba livia}) and several species of bacteria\cite{Johnsen:2008}\cite{Johnsen:2005}\cite{Walker:2003}.

Because of the difficulties involved in locating the magnetic crystals in the bodies of animals there exist experiments that test for the presence of chains of ferromagnetic crystals based on the different results expected from ferromagnetic crystals versus other mechanisms of detecting the magnetic field. A popular such experiment involves training a group of animals to respond behaviourally to external magnetic fields (or using pre-existing such behaviour) and once the behaviour has been established the animals are split into two groups. In the first group small magnets are attached to the animals, near where the researchers believe the magnetic crystals are located in the animal and then the animals are then tested to see if they can still detect the external magnetic fields in their tank or enviroment. The other animals are fitted with brass (non-magnetic) weights of the same weight and dimensions as the magnets and are used as a control group. The theory is that a strong magnet located near the ferromagnetic crystals will be so much stronger than the earth's magnetic field that the magnetic crystals in the animal will align with the field from the bar magnet and any effect from the earth's magnetic field will be too slight to be detected by the animal\cite{Walker:2003}. 

Such experiments have been successfully performed with a range of species including honey bees, sockeye salmon and loggerhead turtles\cite{Walker:2003}\cite{Kirschvink:2001}. Significantly this experiment has been performed on two occasions with elasmobranches, specifically the short-tailed stingray (\textit{Dasyatis Brevicaudata}), by Hodson (2000) and Walker et al (2003). In both experiments the rays were trained to distinguish between the presence or absence of a magnetic intensity anomaly in the experimental tanks. Once the initial conditioning was established the rays were then fitted with small non-magnetic brass weights implanted in the nasal cavity. Again in both experiments the animals were still able to distinguish between the presence and absence of the magnetic anomaly. However when the brass weights were replaced with rare-earth magnets of the same size and weight the animals lost their ability to distinguish between the presence or absence of the magnetic anomaly. Once the magnets were removed the animals were once again able to successfully distinguish between the presence or absence of the magnetic anomaly. While this would seem to support the theory of magnetite-based magnetoreception in elasmobranches it does not, in fact, rule out either of the other two mechanisms proposed for magnetoreception in elasmobranches. A radical-pair based compass would be affected by an attached bar magnet in the same way as a compass based on chains of magnetite crystals. Both would be dominated by the magnetic field of the attached bar magnet and so would lose the ability to distinguish between the field of the anomaly which is considerably further from the sensory organ than the bar magnet. As for the induction-based model, it is true that a magnet attached to an electroreceptor will not induce an electric field as there will be no relative motion between the magnetic field of the attached magnet and the electroreceptor. However the short-tailed stingray possess in the order of one to two thousand ampullae of Lorenzini which are dispersed over a wide area of the ray's body\cite{Montgomery:1999}. While the bar magnet was fixed in the nasal cavity of the animal, the animal was still swimming and so there would have been a considerable amount of movement within the body of the ray. Molteno and Kennedy (2009) showed that for a small magnet (B = 0.02T) relative movement of ~100$\mu$m would be enough to induce a voltage well within the sensitivity of the ampulla. So while the attached magnet did disrupt the animal's ability to distinguish between the presence or absence of a magnetic anomaly we cannot say that this is because it detects magnetic fields via chains of magnetite crystals. 

Another technique that is proposed to demonstrate the existence of a ferromangetic mechanism for navigation is to expose the animal to strong, pulsed magnetic fields (around 500$\mu$T for 5 ms\cite{Johnsen:2008}). The theory is that this strong magnetic field, if aligned antiparallel to the background field, will cause the ferromagnetic crystals within the animals to switch polarization direction and effectively cause the animals to navigate in the opposite direction to prior, i.e. a pigeon flying south instead of flying north. Experiments of this type have successfully performed with bacteria, bees and birds\cite{Kirschvink:2001}. In these experiments there was a controlled application of a weak DC-biasing field in an attempt to counter the induced electric fields from the quickly changing magnetic field\cite{Kirschvink:2001}, however it is possible that the magnetic pulse could have affected some other physiological process which then produced the navigational errors seen in the experiments\cite{Johnsen:2008}. To the author's knowledge though such experiments have not yet been performed with elasmobranches.

\subsection{Radical Pairs}

Theoretical calculations have shown that radical pair recombination may be sensitive enough to magnetic fields that it could be affected by fields near the intensity of the geomagnetic field\cite{Grissom:1995}. An indirect way of testing the optically-pumped radical pair hypothesis that has been performed in experiments is to train birds to orient themselves in response to the magnetic field and then change the frequency of the light in the room so that, according to the theory, there would not be enough energy in the photons to excite the molecule into producing the radical pairs. Experiments of this type have been performed with European robins and Australian silvereyes\cite{Wiltschko:1993}\cite{Wiltschko:1995}. These experiments found that both species of birds could orient themselves (with respect to the earth's magnetic field) under light centered at 443nm (blue) and light centered at 565nm (green) but were disoriented when placed under light centered at 630 nm (red). This could be taken as evidence that light of a certain frequency is necessary to excite the chemical into its radical pairs which would be consistent with the radical pairs hypothesis. However some have questioned whether experiments of this type can account for the possibility that the birds found the red light distressing in completely unrelated physiological processes and these processes, in turn, affect the birds motivation\cite{Johnsen:2005}\cite{Kirschvink:2001}. Also the radical pair lifetime necessary for significant magnetic field effects to build up ($>$ 100ns) is long enough that the radical pair could lose its spin coherence through spin-lattice relaxation or dephasing processes. However tagging records show that blue sharks and scalloped hammerheads can navigate in steady directions at night and at depth\cite{Carey:1990}\cite{Klimley:1993}\cite{Bonfil:2005}, thus presumably ruling out a navigational system exclusively using an optically-pumped radical pair method of magnetoreception for elasmobranch navigation.

\subsection{Induction}

Simple physics says that as a conducting rod \textit{l} moves with a velocity \textit{v} through a magnetic field \textit{B} a potentital difference \textit{V} will be induced between the ends of the rod equal to $lvBsin(\theta)$ where $\theta$ is the angle between \textit{v} and \textit{B}. In essence this is the basis for how an induction-based theory of magnetoreception in elasmobranches would work. Elasmobranches are able to detect minute fluctuations in electric fields around them via their ampullae of Lorenzini.

 Each ampulla essentially consists of a pore that is open to the water around the shark and connects to a canal filled with an electrically conductive jelly and surrounded by a casing of epithelial cells of particularly high resistance (2-3 times higher than the resistivities of other, comparable epithelial cells)\cite{Waltman:1966}. The canal terminates close to the surface of the skin in the inner ampulla which is the electrical receptor of the system\cite{Murray:1974}. Each elasmobranch may have in the order of one to two thousand ampullae of Lorenzini dispersed predominantly around the head.

The electrical sensitivity of elasmobranches is incredibly high, with recorded behavioural changes for field gradients as low as 1nVcm\textsuperscript{-1}\cite{Kajiura:2002}\cite{Kalmijn:1982}\cite{Murray:1974}. It is believed that the electrical sense of elasmobranches is primarily used in short-range prey location. When attracted to an area using a food odour both blue sharks and dogfish preferentially attacked an electric dipole created by two electrodes rather than food buried close by\cite{Kalmijn:1982}. This theory is reinforced as the greatest concentrations of ampullae of Lorenzini exist around the mouth, suggesting that when searching for food, possibly buried, the electric sense acts as a guide for where the shark should attack\cite{Murray:1974}. CITED UP TO HERE! However given the extreme sensitivity of the elasmobranch's electric sense it is possible that it could detect the voltage induced in its body as it swims through the geomagnetic field and use that induced voltage to calculate its heading. M. Paulin(1995) showed quantitatively that a shark with  a modest swimming speed of 0.5ms\textsuperscript{-1} would induce a voltage that was in the order of 16nVcm\textsuperscript{-1} and so well within the threshold of the shark's electric sense. 

\section{More on Induction-based Magnetoreception}
\subsection{AC Voltages}

While this all seems relatively elementary there are several factors which complicate the matter. Chiefly it has been shown that elasmobranches are unable to detect DC voltages via their electric sense. Their bandwidth has the highest response between 0.1-10 Hz and the responses, both neurological and behavioural drop off quickly outside of this range\cite{Montgomery:1999}\cite{Montgomery:1984}. Uniform horizontal motion in the earth's magnetic field would result in a DC induced voltage which the animal would be unable to detect. However M.Paulin(1995) pointed out that as the shark swims horizontally forward it's head swings laterally. Most of the ampullae of Lorenzini in sharks are located in and around the head and so this lateral motion would essentially add a time-varying component to the velocity of the shark. Therefore as well as the induced DC voltage from moving with a horizontal velocity $v$ there will also be an AC component due to the lateral movement of the head which the animal could then detect. The voltage induced in the animal is proportional to the sine of the angle which the shark's head velocity vector makes with magnetic north. This angle is $\psi = \alpha sin(\omega t) + \theta$ where $\alpha$ is the maximum amplitude of the head swing, $\omega$ is the frequency of the head swing and $\theta$ is the horizontal heading of the shark. The voltage induced in the shark is then,

\vspace{4mm}

$V = \lambda vBsin(sin(\omega t) + \theta)$

\vspace{4mm}

There are some instructive cases for $\theta$ that help illustrate how the animal may navigate with the induced voltage. Firstly for $\theta = 0$ i.e. a northerly heading we have no DC offset as the direction of swimming is aligned with the magnetic field lines. If we assume a small $\alpha$ then we can apply the approximation $sin(x) \approx x$ and so we have the induced voltage of the form,

\vspace{4mm}

$V = \lambda \alpha v B sin(\omega t)$

\vspace{4mm}

So the receptor voltage just sinusoidally fluctuates at the swimming frequency and in phase with the head movements. While swimming east (again assuming small $\alpha$) the receptor voltage is $V \approx \lambda vB(1 - \alpha ^2/2sin(\omega t)^2)$. Replacing $sin(\omega t)^2 = (1/2)(1 - cos(2\omega t))$ we see that there is a DC offset of $\lambda vB(1 - \alpha ^2 /4)$ and an AC component that is equal to 

\vspace{4mm}

$V = -(1/4)(\lambda vB\alpha ^2)cos(2\omega t)$  

\vspace{4mm}

Essentially this means that the receptor voltage will be some combination of sine and cosine waves with frequencies of $\omega$ and $2\omega$ respectively. Also the DC offset will vary between $0V$ for a north or south heading and $\pm \lambda vB(1 - \alpha ^2/4)$ for an east or west heading. 

\subsection{Comparison with the vestibular frequency}

One of the major difficulties involved with induction-based magnetoreception is the sizes of the signals involved. In the demonstration case of a shark swimming at 0.5ms\textsuperscript{-1} with head movement of $\alpha = \pi /8$ at $45\circ$S the amplitude of the AC receptor voltage is $\approx 80$nVcm\textsuperscript{-1} when swimming north-south or 16nVcm\textsuperscript{-1} when swimming east-west. To get an idea of the size of these voltages 16nVcm\textsuperscript{-1} is equivalent to a 1.6V battery with one terminal in Dunedin harbour and the other terminal at the Chatham Islands. Then there is the enviromental noise that will cover these signals, for example, as sea water is an ionic solution as the ions in the water move with the current they will experience a force from the magnetic field, this will, in turn induce a voltage in the sea water that can reach several hundred $\mu$V. Even plankton can generate dipole fields of up to 1mV which would add to the noise in the electrosensory system. 

As to how the animal may filter the signal we have already mentioned one important aspect, that is that they are insensitive to DC fields. Their bandwidth of around 0.1-10Hz\cite{Montgomery:1994} means that enviromental fields caused by ionic current flow vary too slowly for there to be much picked up by the animal's electric sense. As for biological fields these are well within the animal's bandwidth but they are relatively short range AC signals \cite{Camperi:2007}, i.e. as soon as the animal has passed around a meter even quite large biological signals ~100$\nu$V will quickly go to zero. However some base level of noise is unavoidable and so some signal processing would be required to unbury the small navigational signal. One way to do this would be to use the animal's vestibular signal as a guide. As the receptor voltage is some sinusoidal signal with frequency of the vestibular signal ($\omega$) and ($2\omega$). By comparing the receptor signal with the vestibular signal and picking out only parts of the receptor signal with frequencies of $\omega$ or $2\omega$ then a great improvement of the signal to noise ratio (SNR) could theoretically be found\cite{Paulin:1995}\cite{Molteno:2009}.



\subsection{Current drift}

One of the great advantages of induction-based magnetoreception versus magnetite-based magnetoreception is the fact that induction works with the shark's total velocity, that is, it's swimming velocity and the velocity of any currents that it is in while swimming. As we have already seen the receptor voltage at the individual ampulla will be of the form $V = Asin(\omega t) + Ccos(2\omega)$ where $A$ and $C$ are the coefficients determined by the bearing of the animal. Let us take a more quantitative look at how a shark could interpret the voltage it detects to resolve a bearing, both without and then with a drift velocity $v^d$. Let the shark's swimming velocity be $v^s = (v^s_x, v^s_y, 0)$ and the geomagnetic field be $B = (0, B_y, 0)$. The receptor electric field for the case with no drif velocity $E^s$ is then 

\vspace{4mm}
$E^s = v^s \times  B,$


      $ = [0, 0, B_y(v^s_x cos(\alpha sin(\omega t)) + v^s_y sin(\alpha sin(\omega t)))],$

\vspace{4mm}

The form for $E^s_z$ above can be written in terms of Bessel functions ($Jn(\alpha)$) like so,

\vspace{4mm}
$E^s_z =  B_y(v^s_x cos(\alpha sin(\omega t)) + v^s_y sin(\alpha sin(\omega t))),$

$\approx B_y(v^s_x J_0 (\alpha) + 2v^s_y J_1 (\alpha) sin(\omega t) + 2v^s_x J_2 (\alpha) cos(2\omega t)),$

$= A_0 + A_\omega sin(\omega t) + A_2\omega cos(2\omega T),$

\vspace{4mm}

Then defining $\tau = A_2\omega / A_\omega$ we see that

\vspace{4mm}
 $\tau = A_2\omega / A_\omega = v^s_x J_2 (\alpha)/v^s_y J_1 (\alpha) = tan(\theta) J_2 (\alpha)/ J_1 (\alpha),$

\vspace{4mm}
Therefore for the case where there is no drift velocity $v^d$ the bearing of the shark can be extracted from the electrical signal with relative ease. However when $v^d \neq 0$ this simple way of extracting the bearing is not sufficient and a nonlinear approach must be taken. With respect to the mechanism through which the animal detects the electrical stimulus a nonlinear model is very acceptable as the repsonses, both behavioural and neurological for voltages in the electrosensory system is highly nonlinear. 

\subsection{Neuronal picture}
An ampulla of Lorenzini is, essentially, a canal filled with an electrically conductive jelly that is open to the ocean at one end and terminates at the electroreceptor itself. Each ampulla of Lorenzini in the shark do no send a continuous voltage signal directly to the brain. Instead the information about the voltage signal is sent via a group of primary afferent neurons around each ampulla, typically around 5-12 neurons per ampulla\cite{Murray:1974}. The neurons send the information to the brain as a series of voltage spikes along the afferent nerve. Each spike is around 10ms long and each neuron emits a resting firing rate of around 34Hz\cite{Camperi:2007}. Then according to the voltage detected across the ampulla the firing rate changes. While the change is nonlinear, in general increasing the firing rate for a negative voltage and decreasing the firing rate for a positive voltage. Camperi et al (2007) fitted a sigmoid function, firing rate(Hz)$ = 1.6 + 62/(1 + 0.9\times $exp$($Vsignal$\times 10^6/11.5))$ to experimental data and this gain function was used in the implementation of the work in this project.

Once the action potentials from all the afferent neurons reach the animal's brain it must first decode them back into some sort of voltage signal in order for it to process the information that they contain. It could potentially do this by analysing the number of spikes being received from the afferent neurons at one particular time ($t$) and from those signals infer an approximate firing rate at time ($t$).  

\section{Building a computational model}
\subsection{Aims}
The main aim of this project is to build a computational model that could essentially perform the kind of calculations that we have inferred the sharks make in order to navigate via the induced voltage from their motion in the geomagnetic field. Firstly we would like to create a system that can take a voltage signal in, encode it as a series of $N$ action potentials for $N$ neurons and then be able to decode those $N$ action potentials into a single voltage signal that resembles the original voltage signal. Once this model has been set up we will compare the fit of the recovered voltage signal to the original for many values of $N$ and see how the fit improves with increasing number of neurons. 

Once we have a model that successfully decodes the $N$ action potentials we will feed it a voltage signal that would correspond to the induced voltage in a shark as it swims and from there try and recover a bearing from that signal. At this point we will also use a modelled vestibular signal to help improve the signal to noise ratio (SNR) as proposed in section 4.2. Initially this will be done with little noise and no drift velocity and once it has been achieved first larger values of noise will be added to bury the signal further and the drift velocity eventually added to test the nonlinear model theorized by Molteno and Kennedy (2009).

When the model is able to retrieve a compass heading for drift velocities the fit (in radians) of this will be analysed against the number of neurons to see if for the number of neurons in the electrosensory system we are able to retrieve a compass heading in the same order of magnitude as that estimated from tagging and behavioural measurements.

%Bibliography
\bibliography{Thesis_Bibliography}{}
\bibliographystyle{plain}
\end{document}
\end{document}